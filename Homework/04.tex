\documentclass[11pt, oneside]{article}
\usepackage[letterpaper, margin=2cm]{geometry}
\usepackage{MATH565}

\begin{document}
\noindent \textbf{\Large{Caleb Logemann \\
MATH 565 Continuous Optimization \\
Homework 4
}}

%\lstinputlisting[language=Python]{H01_23.m}
\begin{enumerate}
  \item % #1
    Let $\M{A} \in \RR^{n \times n}$ be a symmetric positive definite matrix.
    \begin{enumerate}
      \item[(a)]
        Show that the unit vectors $\v{d}_1, \v{d}_2, \ldots, \v{d}n$ are
        $\m{A}$-conjugate vectors if and only if $\M{D}^T \M{A} \M{D} = \M{A}$
        where $\M{D} = \br{\v{d}_1, \v{d}_2, \ldots, \v{d}_n}$.

      \item[(b)]
        If $\M{Q} \in \RR^{n \times n}$ is an orthogonal matrix
        $(\M{Q}^T \M{Q} = \M{I})$, $\v{d}_1, \v{d}_2, \ldots, \v{d_n}$ are
        $\M{A}$-conjugate vectors, and
        $\M{D} = \br{\v{d}_1, \v{d}_2, \ldots, \v{d}_n}$, show that the columns
        of $\M{D} \M{Q}$ are also $\M{A}$-conjugate vectors.
    \end{enumerate}

  \item % #2
    Page 162: Problem 6.3 \\
    Verify that (6.19) and (6.17) are inverses of each other.

  \item % #3
    Page 162: Problem 6.4 \\
    Use the Sherman Morrison formula (A.27) to show that (6.24) is the inverse
    of (6.25).

  \item % #4
    Page 162: Problem 6.6 \\
    The square root of a matrix $A$ is a matrix $A^{1/2}$ such that
    $A^{1/2} A^{1/2} = A$.
    Show that symmetric positive definite matrix $A$ has a square root, and that
    this square root is itself symmetric and positive definite.
    % (Hint: Use the factorization A = UDU^T (A.16), where U is orthogonal and D
    % is diagonal with positive diagonal elements)

  \item % #5
    Implement the BFGS Method (Algorithm 6.1 on Page 140, or lecture notes).

  \item % #6
    Apply the BFGS method to the following function:
    \[
      f(x, y) = (x^2 + y - 11)^2 + (x + y^2 - 7)^2.
    \]
    Use $\M{H}_0 = \M{I}$ and an exact line search for each step length.
    Do all of the following:
    \begin{itemize}
      \item
      \item
      \item
      \item
    \end{itemize}
\end{enumerate}
\end{document}

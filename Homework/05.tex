\documentclass[11pt, oneside]{article}
\usepackage[letterpaper, margin=2cm]{geometry}
\usepackage{MATH565}

\begin{document}
\noindent \textbf{\Large{Caleb Logemann \\
MATH 565 Continuous Optimization \\
Homework 5
}}

%\lstinputlisting[language=Python]{H01_23.m}
\begin{enumerate}
  \item % #1
    Page 269: Problem 10.1 \\
    Let $J$ be an $m \times n$ matrix $m \ge n$.
    \begin{enumerate}
      \item[(a)]
        Show that $J$ has full column rank if and only if $J^T J$ is nonsingular.

        \begin{proof}
          Suppose that $J$ has full column rank, then $J$ has a full singular
          value decomposition.
          That is $J = U \Sigma V^T$, where $U$ is an orthogonal $m \times m$
          matrix, $V$ is an orthogonal $n \times n$ matrix and $\Sigma$ is a
          diagonal $m \times n$ matrix.
          Since $J$ is full column rank this implies that the diagonal of
          $\Sigma$ is nonzero.
          Now using this decomposition $J^T J$ can be written as
          \[
            J^T J = (U \Sigma V^T)^T (U \Sigma V^T) = V \Sigma^T U^T U \Sigma V^T = V \Sigma^T \Sigma V^T.
          \]
          Note that $\Sigma^T \Sigma$ is a $n \times n$ matrix, and it is
          diagonal as it is the product of diagonal matrices.
          Also the diagonal of $\Sigma^T \Sigma$ is nonzero, as the diagonal of
          $\Sigma$ is nonzero.
          Thus this shows that $J^T J$ can be diagonalized as
          $V \Sigma^T \Sigma V^T$ and that the eigenvalues are nonzero.
          Therefore $J^T J$ is nonsingular as the eigenvalues are nonzero.

          Now suppose that $J^T J$ is nonsingular. 
        \end{proof}

      \item[(b)] % Done
        Show that $J$ has full column rank if and only if $J^T J$ is positive definite.

        \begin{proof}
          Suppose that $J^T J$ is positive definite.
          This implies that the eigenvalues of $J^T J$ are all strictly greater
          than zero.
          Since all of the eigenvalues are strictly positive, this implies that
          $\det(J^T J) > 0$ or equivalently that $J^T J$ is nonsingular.
          Now by part (a) this implies that $J$ has full column rank.

          Suppose on the other hand $J$ has full column rank.
          Now by part (a) this implies that $J^T J$ is nonsingular.
          Since $J^T J$ is nonsingular, the eigenvalues of $J^T J$ are nonzero.
          As in part (a) it was shown that $J^T J$ could be diagonalized as
          $V \Sigma^T \Sigma V^T$.
          Considering the diagonal of $\Sigma^T \Sigma$, we see that the
          eigenvalues of $J^T J$ are the squares of the singular values of $J$.
          Since squares are all nonnegative, this shows that the eigenvalues
          of $J^T J$ are nonnegative.
          This fact along with the fact that the eigenvalues are nonzero implies
          that $J^T J$ has only positive eigenvalues or equivalently that $J^T J$
          is positive definite.
        \end{proof}
    \end{enumerate}

  \item % #2
    Page 269: Problem 10.5 \\
    Suppose that each residual function $r_j$ and its gradient are Lipschitz
    continuous with Lipschitz constant L, that is,
    \[
      \abs{r_j(x) - r_j(x^*)} \le L \norm{x - x^*}, \quad \norm{\nabla r_j(x) - \nabla r_j(x^*)} \le L \norm{x - x^*}
    \]
    for all $j = 1, 2, \ldots, m$ and all $x, x^* \in D$, where $D$ is a
    compact subset of $\RR^n$.
    Assume also that the $r_j$ are bounded on $D$, that is, there exists $M > 0$
    such that $\abs{r_j(x)} \le M$ for all $j = 1, 2, \ldots, m$ and all
    $x \in D$.
    Find Lipschitz constants for the Jacobian $J$ (10.3) and the gradient
    $\nabla f$ (10.4) over $D$.

    First I will suppose that all norms are the 2-norm or the matrix norm
    induced by the vector 2-norm.
    First I will consider the Jacobian.
    \begin{align*}
      \norm{J(x) - J(x^*)} &= \sup[\norm{y} = 1]{\norm{(J(x) - J(x^*))y}} \\
      &= \sup[\norm{y} = 1]{\sqrt{\sum*{j = 1}{m}{\p{(\nabla r_j(x) - \nabla r_j(x^*))^T y}^2}}}
      \intertext{Using the Cauchy-Schwarz inequality}
      &\le \sup[\norm{y} = 1]{\sqrt{\sum*{j = 1}{m}{\p{\norm{\nabla r_j(x) - \nabla r_j(x^*)} \norm{y}}^2}}} \\
      &= \sqrt{\sum*{j = 1}{m}{\norm{\nabla r_j(x) - \nabla r_j(x^*)}^2}}
      \intertext{Using the Lipschitz continuity of the gradient}
      &\le \sqrt{\sum*{j = 1}{m}{L^2\norm{x - x^*}^2}} \\
      &= \sqrt{mL^2\norm{x - x^*}^2} \\
      &= \sqrt{m}L\norm{x - x^*} \\
    \end{align*}
    Therefore a Lipschitz constant for the Jacobian is $\sqrt{m}L$.

    Now I will consider the gradient $\nabla f$.

  \item % #3
    Consider the underdetermined linear system $Jx = r$, where
    $J \in \RR^{m \times n}$, $x \in \RR^n$, $r \in \RR^m$, and $m < n$
    (i.e. there are less equations than unknowns).
    Assume that the rank of $J$ is $m$ (i.e., it has full rank).
    There will exist infinitely many solutions.
    The minimum norm solution of $Jx = r$ is the solution closest to the origin,
    which may be regarded as the solution of the constrained optimization problem:
    \[
      \min*[x \in \RR^n]{\norm{x}^2} \quad \text{subject to} \quad Jx = r.
    \]
    \begin{enumerate}
      \item[(a)]
        Use the Lagrange multiplier method, derive the solution to this
        optimization problem:
        \[
          x = J^T (J J^T)^{-1} r
        \]

        \begin{proof}
          The Lagrange multiplier method states that the solution to this
          minimization problem satisfies the following equations,
          \begin{align*}
            \nabla_x \mcL(x, \lambda) &= 0, \\
            c_i(x) &= 0, \forall i \in \mcE, \\
            c_i(x) &\ge 0, \forall i \in \mcI, \\
            \lambda_i &\ge, \forall i \in \mcI, \\
            \lambda_i c_i(x) &= 0, \forall i \in \mcE \cup \mcI,
          \end{align*}
          where $\mcL$ is the Lagrangian.
          In this problem the Lagrangian is
          \[
            \mcL(x, \lambda) = \norm{x}^2 - \lambda^T (Jx - r).
          \]
          There are only equality constraints which are
          \[
            Jx - r = 0
          \]
          The full set of equations for this problem is thus
          \begin{align*}
            \nabla_x (\norm{x}^2 - \lambda^T (Jx - r)) &= 0, \\
            Jx - r &= 0, \\
            \lambda^T (Jx - r) &= 0,
          \end{align*}

        \end{proof}

      \item[(b)]
        Find the minimum norm solution of the $3 \times 5$ system $Jx = r$,
        where
        \[
          J =
          \begin{pmatrix}
            1 & 2 & 0 & 3 & 2 \\
            -1 & -1 & 4 & 2 & 0 \\
            3 & -2 & 2 & 1 & 1
          \end{pmatrix},
          \quad
          r =
          \begin{pmatrix}
            4 \\
            1 \\
            -7
          \end{pmatrix}.
        \]
    \end{enumerate}

  \item % #4
    An important problem in signal processing amounts to finding parameters
    $c_1, c_2, \ldots, c_n$ and $\lambda_1, \lambda_2, \ldots, \lambda_n$ such
    that
    \[
      \sum{k=1}{n}{c_k e^{-\lambda_k t}} \approx f(t),
    \]
    for a given signal function $f(t)$.
    One approach for solving this problem is to formulate a nonlinear least
    squares problems.
    Let
    \[
      x := (x_1, x_2, \ldots, x_{2n}) = (c_1, \ldots, c_n, \lambda_1, \ldots, \lambda_n)
    \]
    be the vector of parameters to be determined.
    Let $s_j = f(t_j)$ be given samples of $f$ for $j = 1,\ldots,2n$ and set
    \[
      r_j(x) = \sum{k=1}{n}{c_k e^{-\lambda_k t_j}} - s_j = \sum{k=1}{n}{x_k e^{-x_{n+k}t_j}} - s_j.
    \]
    We then obtain the parameters as the solution of the nonlinear least squares
    problem:
    \[
      \min*[x \in \RR^{2n}]{\norm{r(x)}^2}.
    \]
    \begin{enumerate}
      \item[(a)]
        Find the general expression for $J(x)$.

        The i, j entry of the Jacobian is defined as
        \[
          J(x)_{ij} = \pd{r_j}{x_i}
        \]
        This value can be computed using the following definition of $r_j$ for
        this problem,
        \[
          r_j(x) = \sum{k=1}{n}{c_k e^{-\lambda_k t_j}} - s_j = \sum{k=1}{n}{x_k e^{-x_{n+k}t_j}} - s_j.
        \]
        The entries are
        \begin{align*}
          J(x)_{ij} &= \pd{r_j}{x_i} \\
          \pd*{\sum{k=1}{n}{x_k e^{-x_{n+k}t_j}} - s_j}{x_i}
          \intertext{If $i \le n$, then there exists $k = i$, so}
          J(x)_{ij} &= e^{-x_{n+i}t_j}
          \intertext{If $i > n$, then there exists $k$ such that $i = n+k$, so}
          J(x)_{ij} &= -t_j x_{i-n} e^{-x_i t_j}
        \end{align*}
        Thus the general definition of the Jacobian by entry is
        \[
          J(x)_{ij} =
          \begin{cases}
            e^{-x_{n+i}t_j} & i \le n \\
            -t_j x_{i-n} e^{-x_i t_j} & i > n
          \end{cases}
        \]
        where $i, j = 1, 2, \ldots, 2n$.

      \item[(b)]
        Let $n = 2$, and
        \[
          t = (0.0, 0.3, 0.6, 0.9) \quad \text{and} \quad s = (2.700, 1.480, 0.819, 0.458).
        \]
        In \MATLAB or \PYTHON, program a Gauss-Newton iteration scheme for this
        problem.
        Apply the scheme the following initial guess:
        \[
          x_0 = (1, 1, 1, 2)
        \]
        and run until convergence.


    \end{enumerate}

  \item % #5
    Page 352: Problem 12.4 \\
    If $f$ is convex and the feasible region $\Omega$ is convex, show that local
    solutions of the problem (12.3) are also global solutions.
    Show that the set of global solutions is convex.
    (Hint: See Theorem 2.5.)

    \begin{proof}
      
    \end{proof}

  \item % #6
    Page 353: Problem 12.13 \\
    Show that for the feasible region defined by
    \begin{align*}
      (x_1 - 1)^2 + (x_2 - 1)^2 \le 2, \\
      (x-1 - 1)^2 + (x_2 + 1)^2 \le 2, \\
      x_1 \ge 0,
    \end{align*}
    the MFCG is satisfied at $x^* = (0, 0)^T$ but the LICQ is not satisfied.

    \begin{proof}
      
    \end{proof}

  \item % #7
    Page 354: Problem 12.18 \\
    Consider the problem of finding the point on the parabola
    $y = \frac{1}{5}(x - 1)^2$ that is closest to $(x, y) = (1, 2)$, in the
    Euclidean norm sense.
    We can formulate this problem as
    \[
      \min*{f(x, y)} = (x - 1)^2 + (y - 2)^2 \quad \text{subject to } (x - 1)^2 = 5y.
    \]
    \begin{enumerate}
      \item[(a)]
        Find all the KKT points for this problem.
        Is the LICQ satisfied?

      \item[(b)]
        Which of these points are the solutions?

      \item[(c)]
        By directly substituting the constraint into the object function and
        eliminating the variable $x$, we obtain an unconstrained optimization
        problem.
        Show that the solutions of this problem cannot be solutions of the
        original problem.
    \end{enumerate}

  \item % #8
    Page 354: Problem 12.21 \\
    Find the maxima of $f(x) = x_1 x_2$ over the unit disk defined by the
    inequality constraint $1 - x_1^2 - x_2^2 \ge 0$.

\end{enumerate}
\end{document}

\documentclass[11pt, oneside]{article}
\usepackage[letterpaper, margin=2cm]{geometry}
\usepackage{MATH565}

\begin{document}
\noindent \textbf{\Large{Caleb Logemann \\
MATH 565 Continuous Optimization \\
Homework 3
}}

%\lstinputlisting[language=Python]{H01_23.m}
\begin{enumerate}
  \item % #1
    Page 100: Problem 4.9. \\
    Derive the solution of the two-dimensional subspace minimization problem in
    the case where $B$ is positive definite.

    \begin{proof}
      
    \end{proof}

  \item % #2
    Page 100: Problem 4.10. \\
    Show that if $B$ is any symmetric matrix, then there exists $\lambda \ge 0$
    such that $B + \lambda I$ is positive definite.

  \item % #3
    Implement the linear conjugate gradient method in \MATLAB or \PYTHON.

  \item % #4
    Page 133: Problem 5.1. (Use your method from the previous problem). \\
    Implement Algorithm 5.2 and use it to solve linear systems in which $A$
    is the Hilbert matrix, whose elements $A_{i, j} = 1/(i + j - 1)$.
    Set the right-hand-side to $b = (1, 1, \ldots, 1)^T$ and the initial point
    to $x_0 = 0$.
    Try dimensions $n = 5, 8, 12, 20$ and report the number of iterations
    required to reduce the residual below $10^{-6}$.

  \item % #5
    Page 133: Problem 5.2. \\
    Show that if the nonzero vectors $p_0, p_1, \ldots, p_l$ satisfy (5.5),
    where $A$ is symmetric and positive definite, then these vectors are
    linearly independent.
    (This result implies that $A$ has a most $n$ conjugate directions.)

    \begin{proof}
      
    \end{proof}

  \item % #6
    Let $n = N^2$.
    Downlonad the \MATLAB file CreateA.m from the course website.
    The correct syntax for calling this code is
    \[
      A = CreateA(N);
    \]
    This creates a matrix of size $N^2 \times N^2$.

    Apply your conjugate gradient method to this problem for various $N$.
    Make a table that records the number of iterations required to achieve a
    reasonable tolerance for $N = 10, 20, 40, 80, 160, 320$.
    You should use the same tolerance in each case.
    How does the number of iterations scale with N?
    What does this tell you about the condition number of $S$ as $N$ varies?
\end{enumerate}
\end{document}
